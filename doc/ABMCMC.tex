\documentclass{article}

\usepackage{arxiv}

\usepackage[utf8]{inputenc} % allow utf-8 input
\usepackage[T1]{fontenc}    % use 8-bit T1 fonts
\usepackage{hyperref}       % hyperlinks
\usepackage{url}            % simple URL typesetting
\usepackage{booktabs}       % professional-quality tables
\usepackage{amsfonts}       % blackboard math symbols
\usepackage{nicefrac}       % compact symbols for 1/2, etc.
\usepackage{microtype}      % microtypography
\usepackage{lipsum}		% Can be removed after putting your text content
\usepackage{amssymb,amsmath}
\usepackage{listings}
\usepackage{graphicx}
\usepackage{subfig}
\usepackage{apacite}

\title{Sampling from the Bayesian poster of an agent-based model given partial observations}

%\date{September 9, 1985}	% Here you can change the date presented in the paper title
%\date{} 					% Or removing it

\author{
  Daniel Tang\\
  Leeds Institute for Data Analytics\thanks{This project has received funding from the European Research Council (ERC) under the European Union’s Horizon 2020 research and innovation programme (grant agreement No. 757455)}\\
  University of Leeds\\
  Leeds, UK\\
  \texttt{D.Tang@leeds.ac.uk} \\
  %% examples of more authors
  %% \AND
  %% Coauthor \\
  %% Affiliation \\
  %% Address \\
}


\begin{document}
\maketitle

\begin{abstract}
abstract
\end{abstract}

% keywords can be removed
\keywords{Data assimilation, Bayesian inference, Agent based model, Integer linear programming, predator prey model}

\section{Introduction}
%##########################################

Suppose we have a closed, convex polyhedron defined as
\begin{equation}
AX = B
\label{polyhedron}
\end{equation}
subject to
\[
x_i \in \mathbb{Z}_{\ge 0}
\]
for all $i$, where $X = x_1...x_n$. And a probability distribution over the space of X
\begin{equation}
P(X) = \prod_i \omega_i^{x_i}
\label{probability}
\end{equation}
Our aim is to sample from valid points (i.e. those that satisfy the constraints) with the given probability distribution.

Suppose we have an oracle, $\Omega(M,Y)$, which (deterministically) returns some $X\in \mathbb{Z}_{\ge0}$ such that $MX=Y$ or returns some null vector if no such solution exists. We can use this oracle to find a solution with a given prefix (or to show that no such solution exists). Suppose we have a prefix vector $P$ of length $m$ so we can write equation \ref{polyhedron} as
\[
\begin{pmatrix}
R|S
\end{pmatrix}
\begin{pmatrix}
V \\
W 
\end{pmatrix}
= B
\]
where $R$ consists of the first $m$ columns of $A$ and $S$ the last $n-m$ columns, then we have
\[
SW = B - RV
\]
so if we let $W = \Omega(S,B-RV)$ then either $W$ is the null vector (and no solution exists with $V$ as prefix) or
\[
AX = A\begin{pmatrix}
V \\
W 
\end{pmatrix}
= B
\]
Given this, we can use the shorthand notation $\Omega_{AB}(V) = X$.

Consider now a tree whose nodes are associated with prefix vectors $V$. Each node is a valid prefix so induces a valid solution $X = \Omega_{AB}(V)$ is a valid solution. The children of a node with prefix $V$ and solution $X$ have prefixes of the form $V_c = (X_q|x)$ where $X_q$ is a prefix of $X$ of length $q$, where $q$ is greater than or equal to the length of $V$ and $x$ is a single element, adding one more dimension to the prefix such that $V_c \ne X_{q+1}$.

The root of the tree is the empty prefix, $\emptyset$, and all valid children of a parent are present in the tree, so the tree contains all valid solutions.

We can make this tree into a Markov chain by defining transition probabilities between nodes. Each node induces a target probability given by equation \ref{probability} on its induced solution. We also define the node's \textit{prefix probability} as 
\[
P(V) = \prod_i \omega_i^{v_i}
\]
and a node's extension probability as
\[
P_e(V) = \omega_m^{v_m}
\]
where $m$ is the length of $V$ and $v_m$ is the $m^{th}$ element of $V$.

Given a node with prefix $V$, children $V'_1...V'_n$ and parent $V_p$, the probability of transition from parent to child $V'_q$ is given by
\[
\tau_{V\rightarrow V'_q} = CP_e(V'_q)
\]
the probability of transition to the node's parent is given by
\[
\tau_{V\rightarrow V_p} = \min\left(\tau_c, \frac{C_pP_e(V)P(X_p)}{P(X)}\right)
\]
and $C$ is a normalisation factor given by
\[
C =  \frac{1 - \tau_{V\rightarrow V_p}}{\sum_q P_e(V'_q)}
\]
given this, the acceptance probability of transition to the $q^{th}$ child is
\[
P_a = \min\left(1, \frac{P(X'_q)\tau_{V'_q\rightarrow V}}{P(X)\tau_{V\rightarrow V'_q}}\right)
\]
there are two regimes here: if
\[
\frac{CP_e(V'_q)P(X)}{P(X'_q)} < \tau_c 
\]
then the acceptance probability is 1.

Acceptance probability of transition to parent is
\[
P_a = \min\left(1, \frac{P(X_p)\tau_{V_p\rightarrow V}}{P(X)\tau_{V\rightarrow V_p}}\right)
\]
which is 1 as long as
\[
\frac{C_pP_e(V)P(X_p)}{P(X)} < \tau_c
\]

In the case that a node is a leaf, then the probability of transition to parent, $\tau_{V\rightarrow V_p}$, must be 1 which means acceptance probability is
\[
\frac{P(X_p)C_pP_e(V)}{P(X)}
\]

%\bibliographystyle{unsrtnat}
%\bibliographystyle{apalike} 
\bibliographystyle{apacite}
\bibliography{references}

\end{document}
