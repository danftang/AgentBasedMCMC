% !TEX root = ABMCMC.tex
\section{Related Research}\label{sec:background}

\paragraph{Intro to DA}

\todo[inline]{Do we need a paragraph to very briefly explain how DA works? Basically predict, update, and the posterior}

\paragraph{Sampling}

\todo[inline]{What are there other approaches designed to allow efficient sampling from the posterior? Why are they not appropriate for ABMs?} 

\paragraph{DA \& ABM}

`Live' (i.e. real-time) agent-based modelling could be transformative for many problems~\cite{swarup_live_2020}. It is unsurprising, therefore, that in recent years a literature on the application of traditional data assimilation (DA) methods to agent-based models (ABMs) has begun to emerge. Although there are a reasonable number of examples of dynamic calibration, where model parameters are re-calibrated based on new data, the specific applications of DA to ABMs are much rarer. Examples of data assimilation using well known DA methods -- such as Particle Filters and variants of the Kalman Filter -- applied to ABMs include modelling:
crime~\cite{lloyd_exploring_2016};
bus routes~\cite{kieu_dealing_2020};
pedestrian dynamics~\cite{wang_data_2015, ward_dynamic_2016, clay_realtime_2020, malleson_simulating_2020};
and population movement~\cite{lueck_who_2019}. 

However, the use of standard DA techniques to ABMs faces a significant challenge. DA algorithms were designed for use in modelling physical systems~\cite{lewis_dynamic_2006}, such as the climate or bodies of water, that are largely driven by partial differential equations. In the short term, models are unlikely to diverge far from the real system, so success has been found with ensemble DA methods that rely on being able to generate plausible model trajectories that support the posterior. With these models it is relatively unlikely that a model trajectory would need to be discarded.
This is not the case with agent-based models, however, where the chance that a single model trajectory might plausibly have created some real-world observations is often 0; in many cases a single decision made by an agent might render the entire trajectory implausible. Hence the application of traditional DA algorithms will result in large numbers of `wasted' model runs that create implausible trajectories that are subsequently refuted by the data. 

\todo[inline]{Need any more? What might an argumentative reviewer complain about?}

% Could go into detail about specific methods but probably not necessary.  E.g. Rules out: rejection sampling (because so many will be rejected), MCMC with Gaussian (because again most proposals will be 0 probability); particle filtering (because particles will end up with 0 probability); Kalman Filters (Gaussian assumption not accuate for ABMs).


\paragraph{Simplex algorithm}

\todo[inline]{Dan mentioned that a lot of the success relies on the implementation of the Simpelx algorithm; need to discuss that here? (maybe not)}

