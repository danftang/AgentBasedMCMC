\documentclass{article}

\usepackage{arxiv}

\usepackage[utf8]{inputenc} % allow utf-8 input
\usepackage[T1]{fontenc}    % use 8-bit T1 fonts
\usepackage{hyperref}       % hyperlinks
\usepackage{url}            % simple URL typesetting
\usepackage{booktabs}       % professional-quality tables
\usepackage{amsfonts}       % blackboard math symbols
\usepackage{nicefrac}       % compact symbols for 1/2, etc.
\usepackage{microtype}      % microtypography
\usepackage{lipsum}		% Can be removed after putting your text content
\usepackage{amssymb,amsmath,amsthm}
\usepackage{listings}
\usepackage{graphicx}
\usepackage{subfig}
\usepackage{apacite}

\newtheorem{theorem}{Theorem}

\title{Sampling from the Bayesian poster of an agent-based model given partial observations}

%\date{September 9, 1985}	% Here you can change the date presented in the paper title
%\date{} 					% Or removing it

\author{
  Daniel Tang\\
  Leeds Institute for Data Analytics\thanks{This project has received funding from the European Research Council (ERC) under the European Union’s Horizon 2020 research and innovation programme (grant agreement No. 757455)}\\
  University of Leeds\\
  Leeds, UK\\
  \texttt{D.Tang@leeds.ac.uk} \\
  %% examples of more authors
  %% \AND
  %% Coauthor \\
  %% Affiliation \\
  %% Address \\
}


\begin{document}
\maketitle

\begin{abstract}
abstract
\end{abstract}

% keywords can be removed
\keywords{Data assimilation, Bayesian inference, Agent based model, Integer linear programming, predator prey model}

\section{Introduction}
%##########################################

Suppose we have a closed, convex polyhedron defined as
\begin{equation}
AX = B
\label{polyhedron}
\end{equation}
subject to
\[
x_i \in \mathbb{Z}_{\ge 0}
\]
for all $i$, where $X = x_1...x_n$. And a probability distribution over the space of X
\begin{equation}
P(X) = \prod_i \omega_i^{x_i}
\label{probability}
\end{equation}
Our aim is to sample from valid points (i.e. those that satisfy the constraints) with the given probability distribution.

\section{Pivot sampling}

\begin{theorem}
For any two extreme positive, integer solutions of $AX_a = B$ and $AX_b = B$ there exists a sequence of positive solutions $X_0...X_N$ such that $X_0 = X_a$, $X_N = X_b$ and there exists a sequence of pivots $\pi_1...\pi_N$ such that pivoting at these points produces the sequence of solutions $X_0...X_N$.
\end{theorem}

\begin{proof}
Start with a full pivot with $X_a$ as a solution. Now let $C$ be the set of events/columns in $X_b$ that are not currently pivoted-in and let $R$ be the set of rows whose currently pivoted columns are not in $X_a \cup X_b$. While there exists a column $j \in C$ and a row $i \in R$ such that $A_{ij} = \pm 1$ pivot on $(i,j)$. Since all rows with pivoted columns not in $X_a$ have $B_i = 0$ then, $i \in R \implies B_i = 0$ so the pivot does not alter the solution.

All remaining events/columns in $C$ must have all rows in $R$ either $0$, $>1$ or $<-1$.

If there is an event/column with all rows in $R$ equal to $0$, then we have found a loop in $X_b-X_a$ which must be positive [proof? no subset of $X_b - X_a$ can, when added to $X_a$ result in a negative number], so we can pivot it in.

So, we need to show that it is impossible that we end up with all multiple occupation numbers outside of $X_a \cup X_b$ for all columns/events left in $C$. For Fermionic ABMs this is necessarily true [proof?].

If we choose a currently pivoted-in event at random and fan out along currently pivoted-in events according to the original constraint matrix for those events, then each row has a "distance" from that point. If we pivot as soon as we reach a row that contains a pivot of an event in $C$, then ...? 

Multiple occupation can only occur for interactions (disregarding predicates) so there must remain only interactions in $C$. Also, if interactions must have only singleton states in the original constraint matrix, there must exist singleton events in the pivoted equivalent. So all these singleton events must be in $X_a \cup X_b$ if all events in $R$ are multiply occupied.

At any point during pivoting, there must exist events in $C$ that 

If a column has multiple occupation on all rows in $R$, there must be at least one member of $C$ which replaces at least some of these multiple occupations, since in the final pivot, these will all be zero. But maybe this too has all multiple occupations in $R$. This can't go on forever though...at some point the event must join to $X_a$ in at least one point.

Consider a branch that has multiple occupation in a column

\end{proof}

(Assume partially observed start/end conditions)

Start with the agent conservation equations.

Find the nearest positive corner points to $X_a$ and $X_b$ (call them $X_a'$ and $X_b'$) and make those the first and last transitions.

Form the tree from sources to sinks where nodes are agent-state/timestep pairs and parents are events in $X_a$ that have the node state as a consequence. Since $X_a$ is an extreme point, this tree must exist. Now add agent-state/timestep nodes in $X_b$ that aren't in $X_a$. Now add events in $X_b$ that join sources from separate trees. The remaining events in $X_b$ are pivots. [we can add one more loop for the root of the tree...but which one?]. 

[Better to think in terms of loops rather than trees?]

%Form the bipartite graph with rows-id's on the left and column-id's on the right with edges where there is a non-zero element in (row,column) in $A$. Add only columns with non-zero values in $X_a'$ of $X_b'$. Find a maximal one-one mapping $\mu$ that contains all non-zero columns in $X_a'$ as a subset (this must exist since $X_a'$ is a corner).

Each edge represents a pivot point. Pivot on each edge in the one-one mapping. The remaining edges $\pi_1...\pi_N$ are pivot points in the sequence.

Since $X_a'$ and $X_b'$ are trees, each $\pi_i$ must be a point where the b-tree joins the a-tree.

Each pivot has the effect of clipping off a subtree and adding the root of the subtree to the children of another node. If all edges are positive edges, then the solution is positive.

Each new branch may "require" other agents to be present at certain space-time points. Since the requirements of an event may only contain events in a previous timestep, there cannot be cycles of requirements. So, if we take the pivot with the earliest time, all its requirements must be fulfilled (otherwise $X_b$ would not be a solution), so we arrange the pivots in chronological order and there's the answer!


%\bibliographystyle{unsrtnat}
%\bibliographystyle{apalike} 
\bibliographystyle{apacite}
\bibliography{references}

\end{document}
